\documentclass[acmsmall]{acmart}

\usepackage[utf8]{inputenc}

%% Some recommended packages.
\usepackage{booktabs}   %% For formal tables:
                        %% http://ctan.org/pkg/booktabs
\usepackage{subcaption} %% For complex figures with subfigures/subcaptions
                        %% http://ctan.org/pkg/subcaption


%%%%%%%%%%%%%%%%%%%%%%%%%%%%%%%%%%%%%%%%%%%%%%%%%%%%%%%%%%%%%%%%%%%

\usepackage{calc}

\usepackage{alltt}
\usepackage{xcolor}
\usepackage{graphicx}
\usepackage{xspace}
\usepackage{pifont} % for \fleur macro

\usepackage{listings}

\lstset{language=Sh,%
   basicstyle=\ttfamily,%
   keywordstyle=\ttfamily,%
   columns=fullflexible,%
   moredelim=[is][\itshape]{\#}{\#},%
   showstringspaces=false,%
   commentstyle={\ttfamily\color{darkred}},%
}
\lstnewenvironment{lstsh}%
{\footnotesize \lstset{language=Sh,keepspaces=true}}
{}

\def\shline{\lstinline[language=Sh, basicstyle=\small\ttfamily]}

\newcommand{\anonymize}[2]{#2}

\newcommand{\coqHTMLBase}{https://querycert.github.io/icfp17}
% \newcommand{\coqHTMLBase}{file:///Users/lmandel/Desktop/qcert}
% \newcommand{\coqBaseModule}{doc/Qcert}
\newcommand{\coqBaseModule}{doc}
% \newcommand{\coqBaseModule}{html/Qcert}
% \newcommand{\coqurl}[3]{\coqHTMLBase/\coqBaseModule.#1.#2.html\##3}
\newcommand{\coqurl}[3]{\coqHTMLBase/\coqBaseModule/#2.html\##3}
\newcommand{\projectname}{Q*cert}

\newcommand{\fleur}{\ding{95}}
\newcommand{\coqtop}{{\href{\coqHTMLBase}{\fleur}}\xspace}
% \newcommand{\coqdef}[2]{{\href{\coqHTMLBase/\coqBaseModule.#1\##2}{\fleur}}\xspace}
\newcommand{\coqdef}[3]{{\href{\coqurl{#1}{#2}{#3}}{\fleur}}\xspace}

\newcommand{\NRAEnv}{\texorpdfstring{NRA$^{\!\mbox{\it e}}$}{NRAe}\xspace}
\newcommand{\NRALambda}{\texorpdfstring{NRA$^{\lambda}$}{NRAλ}\xspace}


%% Colors
\definecolor{darkred}{rgb}{0.55, 0.0, 0.0}

%%% textalltt

\makeatletter
\newcommand*{\textalltt}{}
\DeclareRobustCommand*{\textalltt}{%
  \begingroup
    \let\do\@makeother
    \dospecials
    \catcode`\\=\z@
    \catcode`\{=\@ne
    \catcode`\}=\tw@
    \verbatim@font\@noligs
    \@vobeyspaces
    \frenchspacing
    \@textalltt
}
\newcommand*{\@textalltt}[1]{%
    #1%
  \endgroup
}
\makeatother


%%% ODM
\newcommand{\textblue}[1]{{\color{blue}{#1}}}
\newcommand{\textred}[1]{{\color{darkred}{#1}}}
\newcommand{\odmkw}[1]{\textblue{\textit{#1}}}
\newcommand{\odmagg}[1]{{\color{darkred}{#1}}}

\lstdefinelanguage{jrules}{
  keywords={aggregate,do,emit,false,groupby,insert,new,rule,then,true,update,when},
  numbers=none, columns=flexible, captionpos=b,
  basicstyle={\small\selectfont\ttfamily}, keywordstyle=\bfseries,
  morecomment=[l]{//},
  morecomment=[s]{/*}{*/},
  xleftmargin=0.5cm,
  abovecaptionskip=0em,
  belowcaptionskip=0em
% ,frame=tlbr,framesep=4pt,framerule=0pt
}


%%% Local Variables:
%%% mode: latex
%%% TeX-master: "icfp17"
%%% End:


%%%%%%%%%%%%%%%%%%%%%%%%%%%%%%%%%%%%%%%%%%%%%%%%%%%%%%%%%%%%%%%%%%%

\newcommand{\avi}[1]{{\color{red}#1}}
\newcommand{\jerome}[1]{{\color{orange}#1}}
\newcommand{\louis}[1]{{\color{magenta}#1}}
\newcommand{\martin}[1]{{\color{green}#1}}
\newcommand{\josh}[1]{{\color{brown}#1}}

%%%%%%%%%%%%%%%%%%%%%%%%%%%%%%%%%%%%%%%%%%%%%%%%%%%%%%%%%%%%%%%%%%%

\usepackage{hyperref}
 \hypersetup{
    colorlinks=true,
    linkcolor=black,
    citecolor=black,
    filecolor=magenta,
    urlcolor=blue
}

%%%%%%%%%%%%%%%%%%%%%%%%%%%%%%%%%%%%%%%%%%%%%%%%%%%%%%%%%%%%%%%%%%%

%% Conference information (used by SIGPLAN proceedings format)
%% Supplied to authors by publisher for camera-ready submission
\acmConference[PL'17]{ACM SIGPLAN Conference on Programming Languages}{January 01--03, 2017}{New York, NY, USA}
\acmYear{2017}
\acmISBN{978-x-xxxx-xxxx-x/YY/MM}
\acmDOI{10.1145/nnnnnnn.nnnnnnn}
\startPage{1}


%% Copyright information
%% Supplied to authors (based on authors' rights management selection;
%% see authors.acm.org) by publisher for camera-ready submission
\setcopyright{none}             %% For review submission
%\setcopyright{acmcopyright}
%\setcopyright{acmlicensed}
%\setcopyright{rightsretained}
%\copyrightyear{2017}           %% If different from \acmYear


%% Bibliography style
\bibliographystyle{ACM-Reference-Format}
%% Citation style
%% Note: author/year citations are required for papers published as an
%% issue of PACMPL.
\citestyle{acmauthoryear}   %% For author/year citations



\begin{document}

\title{Readme for reproducibility submission of paper ID 294\\{\normalsize (Handling Environments in a Nested Relational Algebra with Combinators and an Implementation in a Verified Query Compiler)}}


%% Author information
%% Contents and number of authors suppressed with 'anonymous'.
%% Each author should be introduced by \author, followed by
%% \authornote (optional), \orcid (optional), \affiliation, and
%% \email.
%% An author may have multiple affiliations and/or emails; repeat the
%% appropriate command.
%% Many elements are not rendered, but should be provided for metadata
%% extraction tools.

\author{Joshua S. Auerbach}
% \authornote{Emeritus}
% \orcid{nnnn-nnnn-nnnn-nnnn}            %% \orcid is optional
\affiliation{
  % \position{Position1}
  % \department{Department}              %% \department is recommended
  \institution{IBM Research}             %% \institution is required
  \streetaddress{1101 Kitchawan Rd}
  \city{Yorktown Heights}
  \state{NY}
  \postcode{10598}
  \country{USA}
}
\email{joshuaauerbachwatson@gmail.com}   %% \email is recommended

\author{Martin Hirzel}
% \orcid{nnnn-nnnn-nnnn-nnnn}            %% \orcid is optional
\affiliation{
  % \position{Position1}
  % \department{Department}              %% \department is recommended
  \institution{IBM Research}             %% \institution is required
  \streetaddress{1101 Kitchawan Rd}
  \city{Yorktown Heights}
  \state{NY}
  \postcode{10598}
  \country{USA}
}
\email{hirzel@us.ibm.com}   %% \email is recommended

\author{Louis Mandel}
% \orcid{nnnn-nnnn-nnnn-nnnn}            %% \orcid is optional
\affiliation{
  % \position{Position1}
  % \department{Department}              %% \department is recommended
  \institution{IBM Research}             %% \institution is required
  \streetaddress{1101 Kitchawan Rd}
  \city{Yorktown Heights}
  \state{NY}
  \postcode{10598}
  \country{USA}
}
\email{lmandel@us.ibm.com}   %% \email is recommended

\author{Avraham Shinnar}
% \orcid{nnnn-nnnn-nnnn-nnnn}            %% \orcid is optional
\affiliation{
  % \position{Position1}
  % \department{Department}              %% \department is recommended
  \institution{IBM Research}             %% \institution is required
  \streetaddress{1101 Kitchawan Rd}
  \city{Yorktown Heights}
  \state{NY}
  \postcode{10598}
  \country{USA}
}
\email{shinnar@us.ibm.com}   %% \email is recommended

\author{J\'{e}r\^{o}me Sim\'{e}on}
% \orcid{nnnn-nnnn-nnnn-nnnn}            %% \orcid is optional
\affiliation{
  % \position{Position1}
  % \department{Department}              %% \department is recommended
  \institution{IBM Research}             %% \institution is required
  \streetaddress{1101 Kitchawan Rd}
  \city{Yorktown Heights}
  \state{NY}
  \postcode{10598}
  \country{USA}
}
\email{simeon@us.ibm.com}   %% \email is recommended



%% Paper note
%% The \thanks command may be used to create a "paper note" ---
%% similar to a title note or an author note, but not explicitly
%% associated with a particular element.  It will appear immediately
%% above the permission/copyright statement.
% \thanks{with paper note}              %% \thanks is optional
                                        %% can be repeated if necesary
                                        %% contents suppressed with 'anonymous'
% \thanks{An earlier version of this paper was published, in
%   French, in \cite{jfla17}.}

%% Abstract
%% Note: \begin{abstract}...\end{abstract} environment must come
%% before \maketitle command
% \input{abstract}


%% 2012 ACM Computing Classification System (CSS) concepts
%% Generate at 'http://dl.acm.org/ccs/ccs.cfm'.
% \begin{CCSXML}
% <ccs2012>
% <concept>
% <concept_id>10011007.10011006.10011041</concept_id>
% <concept_desc>Software and its engineering~Compilers</concept_desc>
% <concept_significance>500</concept_significance>
% </concept>
% <concept>
% <concept_id>10002951.10002952.10003190.10003192</concept_id>
% <concept_desc>Information systems~Database query processing</concept_desc>
% <concept_significance>300</concept_significance>
% </concept>
% </ccs2012>
% \end{CCSXML}

% \ccsdesc[500]{Software and its engineering~Compilers}
% \ccsdesc[300]{Information systems~Database query processing}
%% End of generated code


%% Keywords
%% comma separated list
% \keywords{query compiler, proof assistant, business rules}  %% \keywords is optional


%% \maketitle
%% Note: \maketitle command must come after title commands, author
%% commands, abstract environment, Computing Classification System
%% environment and commands, and keywords command.
\maketitle

The reproducibility artifact accompanying the SIGMOD 2017 paper
entitled \textit{Handling Environments in a Nested Relational Algebra
  with Combinators and an Implementation in a Verified Query Compiler}
is distributed on github as a branch of the main Q*cert code-base on
which the paper is based.

In order to automate the artifact, it comes with a Docker script
which: (i) sets up a Linux virtual machine, (ii) downloads the
software dependencies, (iii) compile Q*cert, (iv) runs the experiments
and (v) re-compile the paper for comparison with the one in the SIGMOD
2017 proceedings.

The rest of this README provides the steps through the complete
process.

\subsection*{Getting the code}

The reproducibility artifact can either be obtained by cloning the
Q*cert repository using git:
\begin{verbatim}
git clone https://github.com/querycert/qcert.git
cd qcert
git checkout sigmod-repro
\end{verbatim}

or by downloading it as a zip file at:
\begin{verbatim}
https://github.com/querycert/qcert/archive/sigmod-repro.zip
\end{verbatim}

and unzipping it.

\subsection*{Getting the TPC-H Benchmark}

The licensing for the TPC-H benchmarks that we use for evaluation does
not allow redistribution as part of this artifact, so it must be
obtained separately.

Go to the TPC-H download page:
\begin{verbatim}
http://www.tpc.org/tpc_documents_current_versions/current_specifications.asp
\end{verbatim}

and click \verb+Download TPC-H_Tools_v2.17.2.zip+ under Source Code/TPC-H.

Fill out the form and agree to the license.  You will then receive an
email with a link to a individualized page hosting the TPC-H
benchmarks.  Download the benchmarks (you will only be able to
download them once; if you need to do this again, you must fill out
the license form again).

Please accept the suggested filename (which will look something like
\verb+...-tpc-h-tool.zip+, where the leading part is some
individualized string), and save it to the \verb+qcert+ directory (the
top level directory of the code) with the following name:
\verb+tpc-h-tool.zip+ (all lowercases).

\subsection*{Getting Docker}

In order to automatically fetch the dependencies and build the code,
you will need to install Docker, which can be downloaded for your platform at:
\begin{verbatim}
https://www.docker.com/
\end{verbatim}

Once docker is installed you will need to launch it (as any other
application for your system, which will start the Docker daemon). It
will need to be running for the remainder of the process.

\subsection*{Reproduce the paper}

You can now run the full \verb+reproduce-paper.sh+ script from the
\verb+qcert+ directory:
\begin{verbatim}
bash-3.2$ cd qcert
bash-3.2$ ./reproduce-paper.sh 
\end{verbatim}

This will This will create a new docker image (i.e., a virtual
machine) called \verb+qcert:repro+ with all required dependencies, and
use it to build the code, run the experiments, and rebuild the
submitted paper.

It will then copy the rebuilt paper out of the image and put back on
your local file system as:
\begin{verbatim}
qcert/qcert-sigmod-2017-paper-reproduction.pdf
\end{verbatim}

Note that the version of the paper published in sigmod 2017 is
provided for convenience in the Q*cert source as:
\begin{verbatim}
qcert/qcert-sigmod-2017-paper-original.pdf
\end{verbatim}

\subsection*{Explore}

After building the docker image with the
\verb+qcert/reproduce-paper.sh+ script, you can get a shall to that image by running:

\begin{verbatim}
docker run -i -t qcert:repro bash
\end{verbatim}

This can be used to poke around and run the compiler against other
examples, including new examples.

Note that \verb+nano+ is provided in the image for basic editing needs
and \verb+sudo apt-get+ can be used to install other editors as
desired.

\end{document}

